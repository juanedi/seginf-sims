\section{Enunciado}

Se debe implementar una aplicación web que cumpla la función de administrador de identidades,
permitiendo unificar el ABM de usuarios a distintos sistemas, incluyendo definición de roles por
aplicación, cambio de password y login unificado, y genere un registro detallado de las acciones
realizadas.

El master de la información se debe almacenar en una BD, y se debe replicar las partes necesarias en un
árbol LDAP, y en una 2da BD que será utilizada para autenticarse a una aplicación. Se debe mostrar una
aplicación que se autentique vía LDAP, y otra que se autentique a la 2da DB. Además de permitir que un
administrador haga todas las operaciones, se debe permitir que el usuario cambie su clave desde el
administrador de identidades.

La herramienta debe presentar un rol de tipo auditor, que permita ver el registro de todas las
operaciones realizadas, y generar reportes.

En todos los casos, las contraseñas se debe almacenar en forma segura, aunque no necesariamente las
claves se almacenan de la misma manera en los distintos sistemas. Desde el administrador de
identidades, se deben poder establecer criterios de complejidad de las contraseñas, y políticas de
vencimiento.

El sistema debe poder aplicar los cambios en el ldap o en la 2da base de datos en forma diferida. Es
decir, si al momento de hacer el cambio, alguno de estos repositorios esta caído, entonces debe poder
actualizar más tarde, cuando el repositorio vuelva a estar en línea.


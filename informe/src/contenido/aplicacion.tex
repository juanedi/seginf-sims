\section{Descripción de la Solución}

La función principal del sistema de administración de identidades es la administración centralizada de usuarios y roles para diversas aplicaciones. La información generada es enviada a las aplicaciones cliente a través del broker de mensajes.

Por cada acción realizada el sistema registra logs de auditoría que posteriormente pueden ser consultados.
Para el acceso a estas funcionalidades el sistema tiene definido un esquema de roles, practicamente un rol por funcionalidad, que es detallado a continuación. 

El sistema cuenta en un comienzo con un usuario administrador (admin), el mismo tiene asociado todos los roles.

Respecto a la seguridad la aplicación de administración chequea su clave utilizando como hash SHA512, pero al igual que el resto de las aplicaciones, la clave está persistida en la base en todos los hash definidos.

Al expirar una clave se notifica a todas las aplicaciones para que éstas dejen de permitir el acceso a la aplicación
para el usuario. Sin embargo, se sigue permitiendo acceder al administrador para que el usuario pueda cambiar
su clave.

Adicionalmente, se notifica al usuario a los 15 y 7 días de que caduque su clave, asi como también se muestra
una alerta visual en el administrador durante los 15 días antes del vencimiento, con el objetivo de recordar
al usuario que renueve su clave con tiempo.

\subsection{Esquema de roles}

En el sistema de administración de identidades están definidos los siguientes roles:

\begin{itemize}
\item Base: Permite a cualquier usuario cambiar su clave, si bien no es un rol persistido cualquier usuario creado tiene permiso de cambiar su clave
\item Creación de usuarios
\item Creación de aplicaciones: Permite el alta de la aplicación, asignándole un usuario encargado de su configuración
y administración.
\item Administración de aplicación: Este rol de otorga \textit{sobre aplicaciones} al momento de crearlas, y 
define qué usuario tiene permiso para otorgar/revocar acceso y roles específicos a otros usuarios sobre esa 
aplicación. Además, el usuario con este permiso será el encargado de realizar la configuración inicial de la
aplicación (credenciales para el broker de mensajes, configurar esquema de roles de la aplicación, etc.).
\item Politica de Claves: Permite dar de alta o modificar políticas de claves
\item Auditoria: Permite consultar los eventos registrados para la auditoría
\end{itemize}


\subsection{Creación de Usuarios}

Para el alta de usuarios se ingresa usuario, email, nombre y apellido.
El sistema envía un mail al usuario con los datos ingresados más una clave generada aleatoriamente. El rol asociado al usuario para el sistema administrador es el base. 

\subsection{Alta de aplicaciones}
Al dar de alta una nueva aplicación es necesario el alta de la misma en el administrador, la creación de un nuevo
usuario y queues en el broker de mensajes y la configuración en las aplicaciones cliente.
 
\subsubsection{Configuración en el administrador}
En primer lugar le damos el alta ingresando nombre, usuario a cargo y el método de encriptación de clave. Esto
da de alta la aplicación dentro del administrador, y realiza la configuración del broker de mensajes (creación
de queues, usuario, configuración de permisos, etc). Desde este momento, cualquier mensaje que sea generado
para la aplicación será enviado al broker y quedará almacenado a la espera de ser consumido.

En esta instancia el usuario asignado como responsable de la aplicación debe configurar la misma, eligiendo la
clave con la que se conectará al broker de mensajes y cargando el esquema de roles de la aplicación.

En caso de que sea conveniente, se pueden setear roles por defecto, los cuáles son sugeridos automáticamente
cuando se asignan nuevos usuarios a la aplicación.

\subsubsection{Configuración de aplicación cliente}
Cada aplicación debe ahora conectarse con el broker de mensajes, utilizando como nombre de usuario el nombre de
la aplicación tal como se cargó en el administrador, y como contraseña la definida al momento de la configuración
inicial.

El broker debería estar configurado para recibir conexiones seguras en el puerto estándar 5671. Depende de la
implemetación de cada cliente de RabbitMQ cómo se definen los certificados en que se confían. Para aplicaciones
Java, se provee junto con la distribución de la aplicación una biblioteca para facilitar la conexión con el
administrador. En el caso de usar la misma, sólo hace falta especificar en un archivo de propiedades la ruta
a un \textit{trustore jks} que tenga la información del certificado del servidor. Para más detalles ver la
configuración de las aplicaciones demo provistas.

Por convención, los nombres de las queues son del tipo: \texttt{NOMBRE\_APLICACION/TIPO\_MENSAJE}. Los distintos
tipos de mensaje se pueden ver en la especificación del protocolo de mensajes, en las siguientes secciones. Todas
las conexiones deben realizarse al virtual host \textit{sims}.

\subsection{Política de Claves}
La política de claves nos permite definir

\begin{itemize}
  \item longitud de la clave
  \item tipo de caracteres de los cuales debe tener al menos una ocurrencia
  \begin{itemize}
    \item minúsculas (a-z)  
    \item mayúsculas (A-Z)
    \item números (0-9)
    \item caracteres especiales (@\#\$\%)
  \end{itemize}
  \item duración (en cantidad de días)
  \item cantidad de claves diferentes sin repetición
\end{itemize}

Esta política es validada en la creación de usuarios y en los cambios de clave. 

\subsection{Auditoría}

Hay determinados eventos que registran log para la posterior auditoría, en un principio el sistema registra

\begin{itemize}
  \item Cambio de clave
  \item Clave expirada
  \item Creación de un usuario
  \item Creación de una aplicación
  \item Asignación o revocación de acceso de un usuario a una aplicación
  \item Asignación o revocación de un rol a un usuario
  \item Nueva política de complejidad de clave
  \item Cambio de la política de complejidad de claves
\end{itemize}

Para la consulta de estos eventos es necesario ser un usuario con el rol auditor.
Se puede buscar por tipo de evneto, rango de fechas, y aplicaciones y usuarios relacionados con el evento.

\subsection{Protocolo de comunicación con aplicaciones}

La comunicación con las aplicaciones cliente se realiza a través del broker de mensajes. El sistema envía mensajes
por cada evento relacionado a una aplicación. La aplicación cliente, que escucha por medio de un listener, debe
interpretar el mensaje e impactar los cambios como corresponda.
Los mensajes se serializan es strings encodeado en UTF-8 formado por una lista de campos separados
por comas. A su vez, hay campos que son listados entre corchetes de elementos separados por comas. Las fechas
se escriben en formato ISO (yyyy-MM-dd). Se detalla a continuación la composición de los distintos tipos
de mensajes enviados a las aplicaciones, y el nombre de las queues en donde se envían:

\subsubsection{Nuevo Usuario}
\begin{description}
  \item[Queue asociada] \texttt{NOMBRE\_APP/NEW\_USER}
  \item[Descripción del evento] se otorgó acceso a un usuario a la aplicación
  \item[Campos] \ 
    \begin{description}
      \item[username] nombre de usuario
      \item[firstName] primer nombre
      \item[lastName] apellido
      \item[hashType] tipo de hash utilizado para la clave (a modo de control, es el mismo de la aplicación)
      \item[password] clave hasheada
      \item[passwordExpiration] fecha de expiración de la clave
      \item[roles] listado de roles del usuario para la aplicación
    \end{description}
  \item[Ejemplo] \texttt{mtinelli,Marcelo,Tinelli,SHA1,0DPiKuNIrrVmD8IUCuw1hQxNqZc=,2012-11-01,[ROL1]}
\end{description}

\subsubsection{Cambio de Clave}
\begin{description}
  \item[Queue asociada] \texttt{NOMBRE\_APP/CHANGE\_PASSWORD}
  \item[Descripción del evento] cambió la clave de un usuario
  \item[Campos] \ 
    \begin{description}
      \item[username] nombre de usuario
      \item[hashType] tipo de hash utilizado para la clave (a modo de control, es el mismo de la aplicación)
      \item[password] nueva clave hasheada
      \item[passwordExpiration] fecha de expiración de la nueva clave
    \end{description}
  \item[Ejemplo] \texttt{dmaradona,SHA256,jGl25bVBBBW96Qi9Te4V37Fnqchz/Eu4qB9vKrRIqRg=,2013-02-01}
\end{description}

\subsubsection{Vencimiento de clave}
\begin{description}
  \item[Queue asociada] \texttt{NOMBRE\_APP/PASSWORD\_EXPIRED}
  \item[Descripción del evento] expiró la clave de un usuario
  \item[Campos] \ 
    \begin{description}
      \item[username] nombre de usuario
      \item[hashType] tipo de hash utilizado para la clave (a modo de control, es el mismo de la aplicación)
      \item[password] hash de la clave expirada (a modo de control, la aplicación ya lo posee)
    \end{description}
  \item[Ejemplo] \texttt{dmaradona,SHA256,jGl25bVBBBW96Qi9Te4V37Fnqchz/Eu4qB9vKrRIqRg=}
\end{description}


\subsubsection{Remoción de usuario}
\begin{description}
  \item[Queue asociada] \texttt{NOMBRE\_APP/DELETE\_USER}
  \item[Descripción del evento] se revocó acceso a un usuario para la aplicación
  \item[Campos] \ 
    \begin{description}
      \item[username] nombre de usuario
    \end{description}
  \item[Ejemplo] \texttt{mtinelli}
\end{description}

\subsubsection{Cambio de roles}
\begin{description}
  \item[Queue asociada] \texttt{NOMBRE\_APP/CHANGE\_ROLES}
  \item[Descripción del evento] se modificaron los roles de un usuario en la aplicación
  \item[Campos] \ 
    \begin{description}
      \item[roles] listado completo de roles actualizados. incluye los que mantiene y los nuevos.
    \end{description}
  \item[Ejemplo] \texttt{dmaradona,[ROL1,ROL3]}
\end{description}

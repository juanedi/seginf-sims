\section{Requerimientos}

Se describe a continuación el desarrollo de una aplicación web para ser utilizada con el fin de
unificar el ABM de usuarios a distintos sistemas. La aplicación brinda credenciales de login unificada,
cambio de clave, y un registro detallado de actividad accesible para auditoría. También permite modificar
el criterio de complejidad y período de validez de las claves.

La comunicación con el resto de las aplicaciones puede realizarse en forma diferida. Es decir, la información
llega desde el administrador al resto de las aplicaciones recién en el momento que éstas están en línea
y en condiciones de operar. En caso contrario, los cambios se almacenarán en el administrador hasta
que puedan ser enviados.

Era requerimiento que haya una copia master de la información guardada en una base de datos, y que cada
aplicación tenga su propia copia de los datos (almacenada en el formato que se prefiera). De esta forma,
nuestra aplicación haría sólo de ABM de estos datos, pero cada \textit{aplicación cliente} estaría a cargo 
de la autenticación utilizando su copia. A su vez, cada aplicación debería poder utilizar un mecanismo distinto
de almacenamiento de claves.

Sobre los requerimientos iniciales presentados anteriormente, se tomaron algunas decisiones adicionales,
validándose en su momento.

En primer lugar, se decidió modelar un esquema de roles no jerárquico para todas las aplicaciones. Esto no
significa que una aplicación no pueda implementar una jerarquía de roles por su parte, sino que esa jerarquía
lógica no será mantenida en los datos por el administrador: el administrador no conocerá esta jerarquía, y los
roles se asignarán aisladamente a los usuarios sin tener en cuenta su relación con otros roles.

Por otra parte, todos los cambios en los datos se realizan a través del administrador. Éste no deberá encargarse
de sincronizar ningún otro tipo de dato entre las aplicaciones, y cualquier modificación en los datos fluye
desde el administrador hacia las aplicaciones, nunca al revés.

Por último, cuando previamente hablamos de \textit{login unificado} nos referimos a que se utilizarán las mismas
credenciales para autenticarse a todas las aplicaciones manejadas por el administrador, pero no que se sincronizará
ningún estado relacionado a las sesiones de un usuario en las aplicaciones. Tanto la autenticación como el manejo
de sesión es realizada de forma completamente independiente por cada aplicación.

Junto con el administrador de identidades, se presentan dos aplicaciones \textit{demo}, configuradas para 
comunicarse con el mismo. Para mostrar la independencia entre las \textit{aplicación cliente} y el administrador,
estas aplicaciones utilizan distintos \textit{stores} para la información: una usa una base de datos distinta,
mientras que la otra autentica contra un árbol \textit{LDAP}.
